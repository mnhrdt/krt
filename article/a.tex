\documentclass[12pt]{article}                  % base article class
\usepackage[osf,sups]{Baskervaldx}             % baskerville, lining figures
\usepackage[bigdelims,baskervaldx]{newtxmath}  % baskerville math
\usepackage{graphicx}


\setlength{\parindent}{0pt}         % no paragraph indentation
\setlength{\parskip}{7pt}           % spacing between paragraphs


\begin{document}

\begin{center}
    \Huge The Kernel Rank Transform
\end{center}

\bigskip

\section{Definition and properties}

Recall the original definition of the rank transform in a discrete setting
(put reference)

Corresponding definition for a continuous domain (find references)

Propose a generalization using an arbitrary kernel, of which the original
definition is a particular case (the kernel being the indicator function of
a square)

{\bf Definition.}
Let~$u:\mathbf{R}^2\to\mathbf{R}$ be a compactly-supported grayscale image
and~$\kappa:\mathbf{R}^2\to\mathbf{R}$ be a nonnegative function of
integral~$1$.  The~\emph{rank transform of~$u$ with kernel~$\kappa$} is the
image defined by
\begin{equation}
R_{\kappa}(u)(x) = \int\kappa(x-y)\mathbf{1}_{\left[u < u(x)\right]}(y)\mathrm{d}y
\end{equation}
by construction,~$R_\kappa(u)$ is an image that takes values in the
interval~$[0,1]$.

The rank transform can be interpreted as a local contrast equalization of
of~$u$, where the concept of ``local'' is defined by the kernel~$\kappa$.
Below we give the elementary properties of rank transforms and propose some
equivalent definitions and approximations of it.

TODO: write intuitive definition in terms of area percent of the support
of~$\kappa$ is covered by the level sets of~$u$.

\subsection{Equivalent definitions}

We can rewrite the definition above in terms of the heaviside step
function~$H(x)=\mathbf{1}_{[0,+\infty[}(x)$:
\begin{equation}
R_\kappa(u)(x)=
\int
\kappa(x-y)
H\left(u\left(x\right)-u\left(y\right)\right)
\mathrm{d} y
\end{equation}

This expression looks eerily similar to a convolution, but it isn't really.

In fact, it can be written as a convolution in~$(x,u)$ space.  We define the
graph of an image~$u:\mathbf{R}^2\to\mathbf{R}$ as the
function~$U:\mathbf{R}^2\times\mathbf{R}\to\mathbf{R}$ defined by
\[
	U(x,t) = \begin{cases}
		0 & \textrm{if $\ t< u(x)$} \\
		1 & \textrm{if $\ t\ge u(x)$}
	\end{cases}
\]
Then we define the ``kernel''
\[
	K(x,t) = \kappa(x)\delta(t)
\]
And now the convolution of~$U$ and~$K$ is
\[
	(U*K)(\xi,\tau)
	=
	\int
	\int U\left(y,t\right)K\left(\xi-y,\tau-t\right)
		\mathrm{d} y
		\mathrm{d} t
\]
\[
	\quad
	=\int
	U(y,\tau)
	\kappa(\xi - y)
		\mathrm{d} y
\]
\[
	\quad
	=\int\kappa(\xi-y)H(\tau - u(y))
		\mathrm{d} y
\]
Thus
\begin{equation}\label{eq:convolution1}
	R_\kappa(u)(x)=(U*K)(x,u(x))
\end{equation}

Example in dimension 1 and k=interval

Example in dimension 2 (image)

\subsection{Simple (formal) properties}

If~$x$ is the site of the global maximum of~$u$, then~$R_\kappa(u)(x)=1$.
This is also true if~$x$ is a local maximum on a neighborhood defined by the
support of~$\kappa$.

Correspondingly, if~$x$ is the site of the global minimum of~$u$,
then~$R_\kappa(u)(x)=0$.

If the level lines of~$u$ are straight lines (for example, when~$u$ is
affine), and~$\kappa$ is symmetric, then~$R_\kappa(u)=\frac12$.  This also
holds locally (where locality is defined by the support of~$\kappa$).




\subsection{Limit for a constant kernel}

In the limit when~$\kappa$ is constant, the rank transform is the histogram
equalization of~$u$.  To see this, notice that the function
\[
H(\lambda) = \int\mathbf{1}_{[u\ge\lambda]}
\]
is the (reverse) accumulated histogram of~$u$.  This is, the area of the domain
where~$u$ has a value larger than~$\lambda$.  The derivative of~$H$ is thus the
histogram of~$u$.

Notice that this global/local relation is the same as found in local
histogram normalization techniques.
The following two formulations are really similar:

Ref. Sapiro, Caselles,  "Histogram modification via Differential Equations",
formula (21)

Ref. Bertalmío et al. "evidence for the intrinsically nonlinear nature of
receptive fields in vision", formula (10) and next



\subsection{Bilateral}
The accumulated histogram can be ``smoothed'' by using a smooth sigmoid instead
of the discontinuous indicator function.  For example, let~$\sigma$ be a step
function centered at~$0$, we can define the~$\sigma$-smoothed accumulated
histogram of~$u$ by
\[
H_\sigma(\lambda)=\int\sigma(u(x)-\lambda)\mathrm{d}x
\]
Thus, a more general version of the definition for the rank transform is
\begin{equation}\label{eq:generalrank}
R_{\kappa,\sigma}(u)(x)=\int \kappa(x-y)\sigma(u(x)-u(y))\mathrm{d}y
\end{equation}
in the limit when~$\sigma$ is the heaviside step function we recover the
original definition~$R_\sigma(u)$.

Notice that the convolution formalism given in
equation~(\ref{eq:convolution1}) puts the~$\sigma$ and~$\kappa$ smoothings
on an equal footing.  Indeed, by defining
\[
	K_\sigma(x,t)=\kappa(x)\sigma'(t)
\]
we can write formula~(\ref{eq:generalrank}) as
\begin{equation}\label{eq:convolution2}
	R_{\kappa,\sigma}(u)(x)=(U*K_\sigma)(x,u(x)).
\end{equation}

\subsection{Iterated filtering}

The kernel rank transform $R_\sigma$ is an operator that transforms images into images of the same size.
What happens when we compute iterates of this operator?   Is it idempotent?  Is it a semigroup?  Do the iterates explose? (unlikely since they are always images with values on the unit interval).

Some experimental observations with a Gaussian kernel of size
6:

The operator is not idempotent (show experiments with a synthetic and a small real image).

The operator is not a semigroup on the parameter $\sigma$.  We built a list of a few hundred images $R_\sigma(u)$ with finely varying $\sigma$, and we compared them to $R_6(R_6(u))$.  They are all very different, the closet one being about $\sigma=4.4$.

The iterates $R_6^n(u)$ seem to converge to a fixed point.  The fixed point does not look like any $R_\sigma(u)$, the closest one being for~$\sigma=5.4$.

Apparently, we can define a limit operator~$R^\infty_\sigma$ by
\[
R^\infty_\sigma(u) := \lim_{n\to\infty}R^n_\sigma(u)
\]
We do not know whether these limit operators are the kernel rank transdform for a different kernel.   If that was the case, the krt of that kernel  would be idempotent.   

\subsection{Related constructions}

Retinex.  Histogram equalization.  Guided filters.  Integral transforms.
Bilateral filtering.
Morphological rank.  Census.  Non-local laplacians.  Poisson editing of
normalized vectors.  Curvature. ``local contrast equalization''

Contrast-limited adaptive histogram equalization (!)


\subsubsection{Histogram modification via Differential Equations}

Ref. Sapiro, Caselles,  "Histogram modification via Differential Equations",
formula (21)

\section{Implementation}

Discrete implementation for a small support kernel, written in C

{\small
\begin{verbatim}
krt [options] KERNEL [IN [OUT]]

IN : input image (by default, stdin)
OUT : output image of the same size as the input (by default, stdout)
KERNEL : string that specifies the kernel k  Examples:

"file.npy" : the kernel is given by a numeric array, the center is at the
center of the image

"gauss:S" : the kernel is a gaussian of that sigma=S (in pixels)

"disk:R" : disk of radius R

"rectangle:N" : centered rectangle of size (2M+1)x(2M+1), where M=floor(N)

options:

-p 0 : getpixel = 0 outside the original image domain
-p 1 : getpixel = nearest neighbor
-p 2 : getpixel = symmetric extension
... (periodic, etc)

\end{verbatim}
}

Generic implementation as a 3d convolution (consider the problem of
gray-scale sampling!), written in C

{\small
\begin{verbatim}
krt3d [options] KERNEL [IN [OUT]]

IN : input image (by default, stdin)
OUT : output image of the same size as the input (by default, stdout)
KERNEL : string that specifies the kernel k  Examples:

"file.npy" : the kernel is given by a numeric array, the center is at the
center of the image, extended by zero outside of its domain

"gauss:S" : the kernel is a gaussian of that sigma=S (in pixels)

"riesz:S" : riesz kernel of parameter S

options:

-s 255 : gray-scale factor
-n 256 : number of gray-scale bins

future:
-h s : gray-level filtering parameter

\end{verbatim}
}

Verification that both implementations give the same result

Comparison of both implementations (table of running times depending on
kernel size/number of gray levels?)

\section{Examples and applications}


\subsection{Effect of the scale parameter}

show square rank scale-space (using square of varying side)

show gaussian rank scale-space


\subsection{Comparison of different kernels}

maybe show other kernel scalespaces ?

multi-scale kernels (e.g. riesz scale space)

non-isotropic kernels ?
(ref. hirschmuller "7x5")



\section{References}



\end{document}


% vim:set tw=76 filetype=tex spell spelllang=en:
