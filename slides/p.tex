\documentclass[t]{beamer}
%\usepackage[utf8]{inputenc}  % to be able to type unicode text directly
%\usepackage[french]{babel}   % french typographical conventions
\usepackage{inconsolata}     % for a nicer (e.g. non-courier) tt family font
%\usepackage{amsthm,amsmath}  % fancier mathematics
%\usepackage{array} % to fine-tune tabular spacing
\usepackage{bbm} % for blackboard 1

\usepackage{graphicx}        % to include images
%\usepackage{animate}         % to include animated images
\usepackage{soul}            % for colored strikethrough
%\usepackage{bbding}          % for Checkmark and XSolidBrush
\usepackage{hyperref,url}

\colorlet{darkgreen}{black!50!green}  % used for page numbers
\definecolor{term}{rgb}{.9,.9,.9}     % used for code insets

\setlength{\parindent}{0em}
\setlength{\parskip}{1em}


% coco's macros
\def\R{\mathbf{R}}
\def\F{\mathcal{F}}
\def\x{\mathbf{x}}
\def\y{\mathbf{y}}
\def\u{\mathbf{u}}
\def\Z{\mathbf{Z}}
\def\ud{\mathrm{d}}
\DeclareMathOperator*{\argmin}{arg\,min}
\DeclareMathOperator*{\argmax}{arg\,max}
\newcommand{\reference}[1] {{\scriptsize \color{gray}  #1 }}
\newcommand{\referencep}[1] {{\tiny \color{gray}  #1 }}
\newcommand{\unit}[1] {{\tiny \color{gray}  #1 }}

% disable spacing around verbatim
\usepackage{etoolbox}
\makeatletter\preto{\@verbatim}{\topsep=0pt \partopsep=0pt }\makeatother

% disable headings, set slide numbers in green
\mode<all>\setbeamertemplate{navigation symbols}{}
\defbeamertemplate*{footline}{pagecount}{\leavevmode\hfill\color{darkgreen}
   \insertframenumber{} / \inserttotalframenumber\hspace*{2ex}\vskip0pt}

%% select red color for strikethrough
\makeatletter
\newcommand\SoulColor{%
  \let\set@color\beamerorig@set@color
  \let\reset@color\beamerorig@reset@color}
\makeatother
\newcommand<>{\St}[1]{\only#2{\SoulColor\st{#1}}}
\setstcolor{red}

% make everything monospace
\renewcommand*\familydefault{\ttdefault}

% define a font size tinier than tiny
\makeatletter
\newcommand{\srcsize}{\@setfontsize{\srcsize}{5pt}{5pt}}
\makeatother


\begin{document}

\addtocounter{framenumber}{-1}
\begin{frame}[plain,fragile]
\LARGE\begin{verbatim}





     The Kernel Rank Transform




rafa & enric
gtti 21/11/2024
\end{verbatim}
\end{frame}


\begin{frame}
EVERYTHING IN ONE SLIDE\\
=======================

{\bf Definition:}
	\fbox{\(\displaystyle
	\textsc{KRT}_{\kappa,\sigma}(u)(y) := \int \kappa(x-y)\sigma(u(x)-u(y))\ud x
	\)}

{\bf Examples:}

\vfill

\vfill

{\bf Applications:}\\
$\qquad$* image normalization\\
$\qquad$* color balance\\
$\qquad$* cool math\\
$\qquad$* ...?
\end{frame}

\begin{frame}
GOOGLE TRENDS\\
=============

	(the joke about popularity being the opposite of originality, with some
	related graphs of google trends)
\end{frame}


\begin{frame}
OUTLINE\\
=======

\vfill

1. Rank Transform\\
$\ \quad${\color{gray}($\approx$ 5min)}

2. Kernel Rank Transform\\
$\ \quad${\color{gray}($\approx$ 15min)}

\vfill

\end{frame}


\begin{frame}[plain,noframenumbering]%

\vfill
\begin{center}
\Huge
--1--\\Rank Transform
\end{center}
\vfill
\small
\centering
{A classical tool in statistics and image processing}
\end{frame}


% 4. rank transform in statistics
\begin{frame}
THE RANK TRANSFORM IN STATISTICS\\
================================
\end{frame}

% 5. rank transform in image processing (definitions)
\begin{frame}
THE RANK TRANSFORM IN IMAGE PROCESSING\\
======================================
\end{frame}

% 6. visualization of the rank transform of various images and window sizes
\begin{frame}
EFFECT OF THE WINDOW SIZE\\
=========================
\end{frame}

% 7. first observation: small window=curvature, large window=histogram equaliz.
\begin{frame}
LIMIT FOR INFINITELY LARGE OR SMALL WINDOWS\\
===========================================
\end{frame}

% 8. integral formulation of the classical rank transform
\begin{frame}
INTEGRAL FORMULATION OF THE RANK TRANSFORM\\
==========================================
\end{frame}

% 9. generalization: the KRT
\begin{frame}
GENERALIZATION OF THE INTEGRAL FORMULATION\\
==========================================
\end{frame}


\begin{frame}[plain,noframenumbering]%

\vfill
\begin{center}
\Huge
--2--\\Kernel Rank Transform
\end{center}
\vfill
\small
\centering
{the subject of our study}
\end{frame}

% 10. formal definition, common gaussian/heaviside cases, generalizes RT
% 11. first formal properties (extreme cases of the parameters)
% 12. visual exploration of the parameters
% 13. first implementation (c code)
% 14. structural properties
% 15. theorem about contrast invariance being a non-differentiable property
% 16. limit for a large kernel
% 17. limit for a vanishingly small kernel, link with curvature (thm)
% 18. ref. alvarez guichard lions evans
% 19. implementation as a convolution in 3D

% 20. particular cases of the KRT, common operators, applications
%   20.1. retinex
%   20.2. dsm/depth image exploration
%   20.3. bilateral filtering
%   20.4. normalizatoin prior to matching (examples w/ cross-correlation scores)
% 21. comparison with other ``whitening'' transforms
%   21.1. laplacian
%   21.2. gradient direction image
%   21.3. integrated gradient direction image
%   21.4. phase image
%   21.5. x-blurred(x)
%   21.6. x-denoised(x)
% 22. summary of particular cases as a 2d table
% 23. choice of implementation depending on parameters
% 24. comparison with perceptron layers / implementation in npu ?

% 25. colophon: source code of the article and the presentation





% 10. formal definition, common gaussian/heaviside cases, generalizes RT
\begin{frame}
DEFINITION OF THE KRT\\
=====================
\end{frame}

% 11. first formal properties (extreme cases of the parameters)
\begin{frame}
FORMAL PROPERTIES (I)\\
=====================
\end{frame}

\begin{frame}
FORMAL PROPERTIES (II)\\
======================
\end{frame}

% 12. visual exploration of the parameters
\begin{frame}
EXPLORATION OF THE PARAMETERS\\
=============================
\end{frame}

% 13. first implementation (c code)
\begin{frame}
IMPLEMENTATION\\
==============
\end{frame}

% 14. structural properties
\begin{frame}
STRUCTURAL PROPERTIES\\
=====================
\end{frame}

% 15. theorem about contrast invariance being a non-differentiable property
\begin{frame}
CONTRAST INVARIANCE\\
===================
\end{frame}

% 16. limit for a large kernel
\begin{frame}
LIMIT FOR LARGE KERNELS\\
=======================
\end{frame}

% 17. limit for a vanishingly small kernel, link with curvature (thm)
\begin{frame}
LIMIT FOR SMALL KERNELS\\
=======================

	(maybe separate the cases into two or three slides)
\end{frame}

% 18. ref. alvarez guichard lions evans
\begin{frame}
	(ref. alvarez guichard lions evans)
\end{frame}

% 19. implementation as a convolution in 3D
\begin{frame}
INTERPRETATION IN 3D\\
====================
\end{frame}


% 20. particular cases of the KRT, common operators, applications
\begin{frame}
PARTICULAR CASES OF THE KRT\\
===========================
\end{frame}

%   20.1. retinex
\begin{frame}
=============
\end{frame}

%   20.2. dsm/depth image exploration
\begin{frame}
=============
\end{frame}

%   20.3. bilateral filtering
\begin{frame}
=============
\end{frame}

%   20.4. normalizatoin prior to matching (examples w/ cross-correlation scores)
\begin{frame}
=============
\end{frame}

% 21. comparison with other ``whitening'' transforms
\begin{frame}
COMPARISON WITH OTHER ``WHITENING'' TRANSFORMS\\
==============================================
\end{frame}

%%   21.1. laplacian
%\begin{frame}
%=============
%\end{frame}
%
%%   21.2. gradient direction image
%\begin{frame}
%=============
%\end{frame}
%
%%   21.3. integrated gradient direction image
%\begin{frame}
%=============
%\end{frame}
%
%%   21.4. phase image
%\begin{frame}
%=============
%\end{frame}
%
%%   21.5. x-blurred(x)
%\begin{frame}
%=============
%\end{frame}
%
%%   21.6. x-denoised(x)
%\begin{frame}
%=============
%\end{frame}

% 22. summary of particular cases as a 2d table
\begin{frame}
PARTICULAR CASES OF THE KRT\\
===========================
\end{frame}

% 23. choice of implementation depending on parameters
\begin{frame}
CHOICE OF IMPLEMENTATION DEPENDING ON THE PARAMETERS\\
====================================================
\end{frame}

% 24. comparison with perceptron layers / implementation in npu ?
\begin{frame}
COMPARISON WITH A PERCEPTRON LAYER\\
==================================
\end{frame}


% 25. colophon: source code of the article and the presentation
\begin{frame}
COLOPHON\\
========
\end{frame}






\end{document}


% vim:sw=2 ts=2 :
